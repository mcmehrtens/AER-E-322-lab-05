% AER E 361 Mission Report Template
% Spring 2023
% Template created by Yiqi Liang and Professor Matthew Nelson

% Document Configuration DO NOT CHANGE
\documentclass[12 pt]{article}
% --------------------LaTeX Packages---------------------------------
% The following are packages that are used in this report.
% DO NOT CHANGE ANY OF THE FOLLOWING OR YOUR REPORT WILL NOT COMPILE
% -------------------------------------------------------------------

\usepackage{hyperref}
\usepackage{parskip}
\usepackage{titlesec}
\usepackage{titling}
\usepackage{graphicx}
\usepackage{graphviz}
\usepackage[T1]{fontenc}
\usepackage{titlesec, blindtext, color} %for LessIsMore style
\usepackage{tcolorbox} %for references box
\usepackage[hmargin=1in,vmargin=1in]{geometry} % use 1 inch margins
\usepackage{float}
\usepackage{tikz}
\usepackage{svg} % Allows for SVG Vector graphics
\usepackage{textcomp, gensymb} %for degree symbol
\hypersetup{
	colorlinks=true,
	linkcolor=blue,
	urlcolor=cyan,
}
\usepackage{biblatex}
\addbibresource{lab-report-bib.bib}
\usepackage{amsmath}
\usepackage{listings}
\usepackage{multicol}
\usepackage{array}

\usepackage{hologo} %KYR: for \BibTeX
%\usepackage{algpseudocode}
%\usepackage{algorithm}
% This configures items for code listings in the document
\usepackage{xcolor}

\usepackage{fancyhdr} % Headers/Footers
\usepackage{siunitx} % SI units
\usepackage{csquotes} % Display Quote
\usepackage{microtype} % Better line breaks
\usepackage{enumitem} % More control over enumerates

\definecolor{commentsColor}{rgb}{0.497495, 0.497587, 0.497464}
\definecolor{keywordsColor}{rgb}{0.000000, 0.000000, 0.635294}
\definecolor{stringColor}{rgb}{0.558215, 0.000000, 0.135316}
\definecolor{mygreen}{rgb}{0,0.6,0}
\definecolor{mygray}{rgb}{0.5,0.5,0.5}
\definecolor{mymauve}{rgb}{0.58,0,0.82}

\lstdefinestyle{customc}{
  belowcaptionskip=1\baselineskip,
  breaklines=true,
  frame=L,
  xleftmargin=\parindent,
  language=C,
  showstringspaces=false,
  basicstyle=\footnotesize\ttfamily,
  keywordstyle=\bfseries\color{green!40!black},
  commentstyle=\itshape\color{purple!40!black},
  identifierstyle=\color{blue},
  stringstyle=\color{orange},
 }

 \lstset{ %
  backgroundcolor=\color{white},   % choose the background color; you must add \usepackage{color} or \usepackage{xcolor}
  basicstyle=\footnotesize,        % the size of the fonts that are used for the code
  breakatwhitespace=false,         % sets if automatic breaks should only happen at whitespace
  breaklines=true,                 % sets automatic line breaking
  captionpos=b,                    % sets the caption-position to bottom
  commentstyle=\color{commentsColor}\textit,    % comment style
  deletekeywords={...},            % if you want to delete keywords from the given language
  escapeinside={\%*}{*)},          % if you want to add LaTeX within your code
  extendedchars=true,              % lets you use non-ASCII characters; for 8-bits encodings only, does not work with UTF-8
  frame=tb,	                   	   % adds a frame around the code
  keepspaces=true,                 % keeps spaces in text, useful for keeping indentation of code (possibly needs columns=flexible)
  keywordstyle=\color{keywordsColor}\bfseries,       % keyword style
  language=Python,                 % the language of the code (can be overrided per snippet)
  otherkeywords={*,...},           % if you want to add more keywords to the set
  numbers=left,                    % where to put the line-numbers; possible values are (none, left, right)
  numbersep=8pt,                   % how far the line-numbers are from the code
  numberstyle=\tiny\color{commentsColor}, % the style that is used for the line-numbers
  rulecolor=\color{black},         % if not set, the frame-color may be changed on line-breaks within not-black text (e.g. comments (green here))
  showspaces=false,                % show spaces everywhere adding particular underscores; it overrides 'showstringspaces'
  showstringspaces=false,          % underline spaces within strings only
  showtabs=false,                  % show tabs within strings adding particular underscores
  stepnumber=1,                    % the step between two line-numbers. If it's 1, each line will be numbered
  stringstyle=\color{stringColor}, % string literal style
  tabsize=2,	                   % sets default tabsize to 2 spaces
  title=\lstname,                  % show the filename of files included with \lstinputlisting; also try caption instead of title
  columns=fixed                    % Using fixed column width (for e.g. nice alignment)
}

\lstdefinestyle{customasm}{
  belowcaptionskip=1\baselineskip,
  frame=L,
  xleftmargin=\parindent,
  language=[x86masm]Assembler,
  basicstyle=\footnotesize\ttfamily,
  commentstyle=\itshape\color{purple!40!black},
}

\lstset{escapechar=@,style=customc}

\titlelabel{\thetitle.\quad}

% From here on out you can start editing your document
\newcommand{\subtitle}[1]{%
  \posttitle{%
    \par\end{center}
    \begin{center}\LARGE#1\end{center}
    \vskip0.5em}%
}

\newcommand{\etal}{\textit{et al}., }
\newcommand{\ie}{\textit{i}.\textit{e}., }
\newcommand{\eg}{\textit{e}.\textit{g}., }

\newcommand\numberthis{\addtocounter{equation}{1}\tag{\theequation}}

% Define the headers and footers
\setlength{\headheight}{70.63135pt}
\geometry{head=70.63135pt, includehead=true, includefoot=true}
\pagestyle{fancy}
\fancyhead{}\fancyfoot{} % clears the headers/footers
\fancyhead[L]{\textbf{AER E 322}}
\fancyhead[C]{\textbf{Aerospace Structures Pre-Laboratory}\\
			  \textbf{Lab 5 Beam Deflection and Analysis}\\
			  Section 4 Group 2\\
			  Matthew Mehrtens\\
			  \today}
\fancyhead[R]{\textbf{Spring 2023}}
\fancyfoot[C]{\thepage}

\begin{document}
\textit{Apply the basic beam theory, which you learned in E M 324 or the like, to directly solve for the deflections for the cantilever beam setup as shown in Fig. 1. Please note that, in the following references, ``$x$ grams load'' or simply ``$x$ grams'' should be interpreted as the ``gravitational weight/force of $x$ grams mass''. Here the quantity of $x$ purely measures the mass itself.}

\section*{Question 1} \label{sec:question_1}
\textit{(30 pts) Derive a general formula by using either method of superposition or discontinuity function method and show the following expression for deflections at $\frac{1}{3}$ beam length measured from the free end, }i.e.\textit{,}
\begin{align*}
	x&=\frac{2}{3}L\\
	\nu&=-\frac{w}{EI}\frac{14}{81}L^3 \numberthis\label{eqn:deflection}
\end{align*}

From the \textit{Cantilevered Beam Slopes and Deflections} table on slide \num{14} of the lecture slides, we are given
\begin{align} \label{eqn:superposition}
	\nu&=-\frac{wx^2}{6EI}(3L-x)
\end{align}
As described, we let $x=\frac{2}{3}L$ and simplify Equation \ref{eqn:superposition}:
\begin{align*}
	\nu&=-\frac{w(\frac{2}{3}L)^2}{6EI}\left(3L-\frac{2}{3}L\right)\\
	\nu&=-\frac{w\frac{4}{9}L^2}{6EI}\left(\frac{7}{3}L\right)\\
	\nu&=-\frac{w\frac{28}{27}L^3}{6EI}\\
	\nu&=-\frac{w}{EI}\frac{28}{162}L^3\\
	\nu&=-\frac{w}{EI}\frac{14}{81}L^3
\end{align*}

\section*{Question 2} \label{sec:question_2}
\textit{(20 pts) Calculate the deflections in \unit{\milli\meter} at $\frac{1}{3}$ beam length measured from the free end for three beams of rectangular cross section subjected to loads at the free end. Let the beam length be \qty{90}{\centi\meter} and use $E=\qty{68.9}{\GPa}$ (for Aluminum 6061-T6511 alloy). The beam cross sections and loads are as follows:}

\begin{enumerate}[label=(\Alph*)]
	\item \textbf{\qtyproduct{12.8 x 6.4}{\milli\meter} (\qty{12.8}{\milli\meter} is the base and \qty{6.4}{\milli\meter} is the height) rectangular cross-section with \qty{100}{\gram} load.}
	
	We are given $L=\qty{0.90}{\m}$, $E=\qty{68.9e9}{\Pa}$, $b=\qty{0.0128}{\m}$, and $h=\qty{0.0064}{\m}$. We can calculate $w$ as shown below
	\begin{align*}
		w&=(\qty{100}{\g})\left(\frac{\qty{1}{\kg}}{\qty{1000}{\g}}\right)\left(\qty[per-mode=fraction]{9.81}{\m\per\s\squared}\right)\\
		w&=\qty{0.9810}{\N}
	\end{align*}
	and we can calculate $I$ as follows
	\begin{align*}
		I&=\frac{1}{12}(\qty{0.0128}{\m})(\qty{0.0064}{\m})^3\\
		I&=\qty{2.796e-10}{\m^4}
	\end{align*}
	Substituting these values into Equation \ref{eqn:deflection}, we find that
	\begin{align*}
		\nu_A&=-\frac{\qty{0.9810}{\N}}{(\qty{68.9e9}{\Pa})(\qty{2.796e-10}{\m^4})}\frac{14}{81}(\qty{0.90}{\m})^3\\
		\nu_A&=\qty{-6.416}{\mm}\\
		\text{or}\\
		\nu_A&=\qty{6.416}{\mm}\downarrow
	\end{align*}
	\item \textbf{\qtyproduct{6.4 x 12.8}{\milli\meter} rectangular cross-section with \qty{200}{\gram} load.}
	
	We are given $L=\qty{0.90}{\m}$, $E=\qty{68.9e9}{\Pa}$, $b=\qty{0.0064}{\m}$, and $h=\qty{0.0128}{\m}$. We can calculate $w$ as shown below
	\begin{align*}
		w&=(\qty{200}{\g})\left(\frac{\qty{1}{\kg}}{\qty{1000}{\g}}\right)\left(\qty[per-mode=fraction]{9.81}{\m\per\s\squared}\right)\\
		w&=\qty{1.962}{\N}
	\end{align*}
	and we can calculate $I$ as follows
	\begin{align*}
		I&=\frac{1}{12}(\qty{0.0064}{\m})(\qty{0.0128}{\m})^3\\
		I&=\qty{1.119e-09}{\m^4}
	\end{align*}
	Substituting these values into Equation \ref{eqn:deflection}, we find that
	\begin{align*}
		\nu_A&=-\frac{\qty{1.962}{\N}}{(\qty{68.9e9}{\Pa})(\qty{1.119e-09}{\m^4})}\frac{14}{81}(\qty{0.90}{\m})^3\\
		\nu_A&=\qty{-3.208}{\mm}\\
		\text{or}\\
		\nu_A&=\qty{3.208}{\mm}\downarrow
	\end{align*}
	\item \textbf{\qtyproduct{12.8 x 12.8}{\milli\meter} square cross-section with \qty{500}{\gram} load.}
	
	We are given $L=\qty{0.90}{\m}$, $E=\qty{68.9e9}{\Pa}$, $b=\qty{0.0128}{\m}$, and $h=\qty{0.0128}{\m}$. We can calculate $w$ as shown below
	\begin{align*}
		w&=(\qty{500}{\g})\left(\frac{\qty{1}{\kg}}{\qty{1000}{\g}}\right)\left(\qty[per-mode=fraction]{9.81}{\m\per\s\squared}\right)\\
		w&=\qty{4.905}{\N}
	\end{align*}
	and we can calculate $I$ as follows
	\begin{align*}
		I&=\frac{1}{12}(\qty{0.0128}{\m})(\qty{0.0128}{\m})^3\\
		I&=\qty{2.237e-09}{\m^4}
	\end{align*}
	Substituting these values into Equation \ref{eqn:deflection}, we find that
	\begin{align*}
		\nu_A&=-\frac{\qty{4.905}{\N}}{(\qty{68.9e9}{\Pa})(\qty{2.237e-09}{\m^4})}\frac{14}{81}(\qty{0.90}{\m})^3\\
		\nu_A&=\qty{-4.010}{\mm}\\
		\text{or}\\
		\nu_A&=\qty{4.010}{\mm}\downarrow
	\end{align*}
\end{enumerate}

\end{document}
